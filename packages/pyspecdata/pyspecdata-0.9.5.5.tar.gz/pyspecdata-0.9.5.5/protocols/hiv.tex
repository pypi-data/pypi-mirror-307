\subsection{Wild-type $T_{1,0}$}\label{sec:florida_sample2_t10}
\timeblockstart
\timeblocktotal{3.6}
This sample says ``bsi quenched,'' so I think it's the wild type $T_{1,0}$.
\subsubsection{initial setup}\maxminutes{57}
\paragraph{always}
Be sure to change section title and label.

\sout{ Copy section label to list for the day. }
Leave samples to thaw on the way over.

\paragraph{capillary prep}
Put spectrometer into tune mode, and turn down power.

Load 3.5\uL of sample.

Press twice in critoseal.
Score other end very lightly.
Break + press twice in semi-soft (clay consistency) wax.
Melt wax.
Clean w/ razor.
\sout{ Inspect + measure with magnifying glasses. }

Put into tune mode before inserting sample, and turn down power.

\subsubsection{set up DNP}\maxminutes{38}
\paragraph{insert probe and autotune}

Turn off field.

Check marker marks on probe.

Insert probe with sample.

Attach tuning box, then position and tighted screw.

Turn up air to 10 SCFH.

Press autotune.

\paragraph{Tune and set resonant field}
Detach mod coil and hang in the plastic loop (here so that I don't potentially mess up microwave tune later!).

Copy nmr template dataset.
\fn{hiv_wt_t10_120301}

Overwrite exp 1 with desired parameters.
\fn{hiv_120301}

Round nearest 0.0005\GHz to get YIG frequency.

Turn on field.

Open first experiment and set NMR sfo1 to (1.5167)*(YIG frequency).

Tune NMR with {\tt wobb}.

\precaution{Run jf\_dnpconf if necessary -- specifically may need to decrease my attenuation settings by 2.7 $dB$ (or just round it to 3 $dB$).}
Tune microwave by hand
{\small (Simultaneously set:
\begin{itemize}
    \item bias (diode centered at 50 $dB$)
    \item frequency (AFC)
    \item phase (coarse: dip in tune mode; fine: furthest right diode w.r.t.signal phase while AFC remains locked)
\end{itemize}
up to 5 $dB$.)}
\begin{lstlisting}
calcfielddata(9.789856,'3MP','cnsi')
\end{lstlisting}

\precaution{If diode is jittery at high powers, set AFC DC gain to 1/10 rather than 1; if that doesn't work, try decreasing the AFC mod amp, etc.}

Check that YIG frequency is still good

EPR to standby.

Set field.

\paragraph{ Match NMR + YIG frequencies}
jf\_zg $\Rightarrow$ for 3.5 $\mu L$ sample, peak should 10-15 high.

Check that the source is on, and reads 0.26 $A$ (rather than 0.54 $A$).

Click lightning bolt to set o1 (while zoomed in with top window).

Set YIG frequency.

Be sure the ppt value is set to the value above.

Repeat jf\_zg to verify that field has stabilized (peak should be at 0).

\precaution{Test amp switch with jf\_setmw (leave other parameters as default).}
Iteratively adjust the field.
{\small
\begin{enumerate}
	\item jf\_setmw (31.5 $dB$, YIG frequency),
	\item then if field offset is significant, adjust $B_0$, otherwise stop
	\item jf\_zgm
	\item then set sfo1 to NMR resonance (lightning bolt) + start over
\end{enumerate}
}

\paragraph{90 time and ready amp}
jf\_zg + zoom + dpl1 + run {\tt paropt} (p1,8,1,3)

\precaution{This should go down to up at the full cycle;
if problems, (p1,1,1,11).}
\precaution{If you want to decrease the narrow noise spikes, minimize WinEPR + turn off bridge + put all bridge cables on top of dewar while paropt is running; could be obsolete with box?}
Flip waveguide switch.

Set + record $t_{90}$ (``\texttt{p1}'')=(length of full cycle [\us])/4 (At this point, you've determined the resonance frequency, resonant field, and $t_{90}$).

\paragraph{run DNP}
Start jf\_dnp: Set minimum and maximum $T_1$ values based on $T_1$ estimate (with heating).

Set number of $T_1$ time to 14~min for this sample, because I'm worried about lifetime issues.

Set watch timer for experiment time.

Cool all samples.

\subsubsection{wait for DNP to run}\maxminutes{90}
\subsubsection{process DNP + put back system}\maxminutes{6}
\paragraph{process}
Transfer NMR files.

Change name, chemical names, concentration, and run number.

For $T_{1,0}$, set dontfit to True, otherwise False.

Run processing.

Mask/unmask $T_1$'s as necessary.

Check that my longest $T_1$ falls within range.

Add to the compilation, to see how the data looks.

Check the consistency of the enhancements with decreasing power.

Redo experiment 5 as 505 \texttt{re 5 1;Ctrl-N} + enter exp 505 +\texttt{zg}.

Comment {\tt search\_delete\_datanode}.


\begin{scriptsize}
\begin{lstlisting}
import textwrap
# re-run
# different initial guess
name = 'hiv_wt_t10_120301'
path = DATADIR+'cnsi_data/'
search_delete_datanode('dnp.h5',name)
# leave the rest of the code relatively consistent
#{{{ generate the powers for the T1 series
print 'First, check the $T_1$ powers:\n\n'
fl = []
t1_dbm,fl = auto_steps(path+name+'/t1_powers.mat',
    threshold = -35,t_minlength = 5.0*60,
    t_maxlen = 40*60, t_start = 4.9*60.,
    t_stop = inf,first_figure = fl)
print r't1\_dbm is:',lsafen(t1_dbm)
lplotfigures(fl,'t1series_'+name)
print '\n\n'
t1mask = bool8(ones(len(t1_dbm)))
# the next line will turn off select (noisy T1
# outputs) enter the number of the scan to remove --
# don't include power off
#t1mask[-1] = 0
#}}}
dnp_for_rho(path,name,integration_width = 160,
        peak_within = 500, show_t1_raw = True,
        phnum = [4],phchannel = [-1],
        t1_autovals = r_[2:2+len(t1_dbm)][t1mask],
        t1_powers = r_[t1_dbm[t1mask],-999.],
        power_file = name+'/power.mat',t_start = 4.6,
        chemical = 'hiv_pr_wt',
        concentration = 0.0, extra_time = 6.0,
        dontfit = True, run_number = 120301,
        threshold = -50.)
standard_noise_comparison(name)
# tried to fix error on cov more fix more fix more fix more fix more fix
\end{lstlisting}
\end{scriptsize}

\paragraph{put back system}
\subparagraph{Always}
Field off.

Hook up mod coil. 

Flip back ESR switch.

Turn off the air + check rate.

Pull out + inspect + measure sample\sout{  with magnifying glasses }.

\subsubsection{wrap up}\maxminutes{18}
SVN, then copy working copy of notebook into compilation.

Clear + copy to protocol.

\subparagraph{Run ESR background scan}
Autotune + run ESR background scan.

\subparagraph{Last experiment only}
If done, remove air tube.

Unscrew top collet + carry box w/ dewar around to back.

Remove glass tube and replace dewar.

Insert inner collet.

Screw down top collet holder.

Cap cavity.

Tell software that bridge is on + switch to tune mode

Check for dip near 9.88~GHz (if without dewar) or 9.31~GHz (if with dewar).

Autotune + record Q.

Remove tuning box.

If done, magnet off, ESR off, chiller off, source output off.

Be sure to take coolers and dewar back to lab.

\timeblockend
