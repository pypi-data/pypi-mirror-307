\subsection{redo 32\mM DOPC}\maxtime{3}\label{sec:concissues_dopc32mMredo}
Here, I am redoing the DOPC high concentration sample (after extrusion problems).

\subsubsection{initial setup}\maxminutes{57}
Be sure to change section title and label !!!

\precaution{Check that copper plate is set up}
\paragraph{sample prep}
Calculate preparation

Prepare sample.

\paragraph{capillary prep}
Prepare capillary with 3.5 $\mu L$ of sample.

\hidden{
\paragraph{On + set up for (first exp only)}
EPR and source on, magnet and air off.

Be sure mod coil is hooked up, and the waveguide switch is set to ESR.

Remove top collets on dewar.

Take box around back.

\na{Remove dewar.}




Insert air tube (put only bottom collet on, tighten very loosely, then add and tighten top collet).

Zip tie air tube.

Attach probe to plate.

Wait until bridge stops flashing, then open WinEPR + switch to tune mode 40~dB.

}
\hidden{ \subsubsection{Run time course} Copy the template.  \fn{popemx_4mM_5p_mmsl_timecourse_110711} Copy first experiment from last one.  \fn{popeca_4mM_5p_mmsl_110708} Turn on the amplifier, with the source off.  Copy experiment 520 from last time into the new folder and remove raw data.  Assume a ninety time of 2.0 $\mu s$ in both exp 1 and 520.  Start power meter program running so that it won't quit.  Set EPR to tune mode at low power.  Mix 1.75 of each sample (mark exact time here).  Load sample.  Turn air up to 10 SCFM.  Tune the EPR quickly by hand, record frequency.  Set to standby and set field.  Tune NMR (forgot this), then check EPR tune.  Use jf setmw to set the frequency and zero in on the signal in exp 1.  Flip EPR switch, remove mod coil, turn on source.  Use jf setmw to set to 6 $dB$ Copy sfo1 to new experiment and start experiment.  Copy data and process when done.  \fn{popemx_4mM_5p_mmsl_timecourse_110711.mat} Since I may see a slight increase, iexpno, and run for twice as long.  
\begin{tiny}
\begin{lstlisting}
fl = []
print 'Next, process the saturation-recovery data:\n\n'
sat_data,fl = integrate(DATADIR+'cnsi_data/popemx_4mM_5p_mmsl_timecourse_110711/',
    r_[520],
    first_figure = fl,
    integration_width = 200,
    phnum = [4], phchannel = [-1],
    dimname = r'exp #',
    pdfstring = 'sat')
sat_data2,fl = integrate(DATADIR+'cnsi_data/popemx_4mM_5p_mmsl_timecourse_110711/',
    r_[521],
    first_figure = fl,
    integration_width = 200,
    phnum = [4], phchannel = [-1],
    dimname = r'exp #',
    pdfstring = 'sat')
nextfigure(fl,'integral')
plot(sat_data,label = 'sat-rec')
plot(sat_data2,label = 'sat-rec 2')
ax = gca()
ylims = array(ax.get_ylim())
ylims[ylims.argmin()] = 0
ax.set_ylim(ylims)
autolegend()
lplotfigures(fl,'timecourse_110708.pdf')
\end{lstlisting}
\end{tiny}
}
\hidden{

\subsubsection{ESR}\maxminutes{10}
Put into tune mode, $40\;dB$ first.

Load sample in the teflon sample holder (I want ~77 $mm$ from collet to bottom of sample, need to remeasure).

Move to near 9.77 and autotune.

Open ESR parameter set similar to this one.
\fn{dopc_32mM_3p_120117.par}

Turn on field.

Copy experiment.

If not exactly the same type of sample, test run to check some stuff.

\na{Only if the experiment says ``uncalibrated'' at the top $\Rightarrow$ ``I'' (interactive spectrometer control) icon, click calibrated, then set parameters to spectrum, then window.}











Stop at second peak.

Check that modulation amplitude is $<$ 0.2 x smallest feature.

Set RG with box.

Check that resolution along $x$ is OK.

Run actual scan.

Save ESR.
\fn{dopc_dil_3p_120118}

For 8 scans, alarm for about 2.5 min.

Cntrl-S after scan is finished.

Hit Cntrl-A for ssh transfer and process.

\precaution{Also remove sample and retune to run the background, if necessary.}


\begin{tiny}
\begin{lstlisting}
dir = DATADIR+'cnsi_data'
files = ['background_120117','dopc_32mM_3p_120117','dopc_dil_3p_120118']
normalize_field = True
find_maxslope = True
subtract_first = True # only because it won't broadcast to the same shape
normalize_peak = True
# the next part should not change, and should be compiled into a function later
dir = dirformat(dir)
fl = figlistl()
if subtract_first:
    firstdata = load_indiv_file(dir+files.pop(0))
legendtext = list(files)
for index,file in enumerate(files):
   data = load_indiv_file(dir+file)
   if subtract_first:
       data -= firstdata
   field = r'$B_0$'
   neworder = list(data.dimlabels)
   data.reorder([neworder.pop(neworder.index(field))]+neworder) # in case this is a saturation experiment
   data -= data.copy().run_nopop(mean,field)
   fl.next('epr')
   v = winepr_load_acqu(dir+file)
   if index == 0:
        fieldbar = data[field,lambda x: logical_and(x>x.mean(),x<x.mean()+10.)]
        fieldbar.data[:] = 0.5
        fieldbar.data[0] = 0.6
        fieldbar.data[-1] = 0.6
        fxaxis = fieldbar.getaxis(field)
   xaxis = data.getaxis(field)
   centerfield = None
   if normalize_field:
       xaxis /= v['MF']
       if index == 0:
           fxaxis /= v['MF']
       newname = r'$B_0/\nu_e$'
   elif find_maxslope:
       deriv = data.copy()
       deriv.run_nopop(diff,field)
       deriv.data[abs(data.data) > abs(data.data).max()/10] = 0 # so it doesn't give a fast slope, non-zero area
       deriv = abs(deriv)
       deriv.argmax(field)
       centerfield = mean(xaxis[int32(deriv.data)])
       xaxis -= centerfield
       if index == 0:
           fxaxis -= centerfield
       newname = r'$\Delta B_0$'
   else:
       newname = field
   data.rename(field,newname)
   if index == 0:
       fieldbar.rename(field,newname)
   mask = data.getaxis(newname)
   mask = mask > mask[int32(len(mask)-len(mask)/20)]
   snr = abs(data.data).max()/std(data.data[mask])
   integral = data.copy()
   integral.data -= integral.data.mean() # baseline correct it
   integral.integrate_cumulative(newname)
   fl.next('epr_int')
   plot(integral,alpha=0.5,linewidth=0.3)
   pc = plot_color_counter()
   integral.integrate_cumulative(newname)
   fl.next('epr')
   if normalize_peak:
      normalization = abs(data).run_nopop(max,newname)
      data /= normalization
   ax = gca()
   plot(data+array(ax.get_ylim()).min(),alpha=0.5,linewidth=0.3)
   axis('tight')
   if index == 0:
       fieldbar *= array(ax.get_ylim()).max()
   if centerfield != None:
      legendtext[index] += ', %0.03f $G$'%centerfield
   legendtext[index] += r', SNR %0.2g $\int\int$ %0.3g'%(snr,integral[newname,-1].data[-1])
#xtl = ax.get_xticklabels()
#at.xaxis.tick_top()
#map( (lambda x: x.set_visible(False)), xtl)
plot(data.getaxis(newname)[mask],zeros(shape(data.getaxis(newname)[mask])),'k',alpha=0.2,linewidth=10)
fl.next('epr')
plot(fieldbar,'k',linewidth = 2.0)
#autolegend(legendtext)
axis('tight')
fl.next('epr_int')
autolegend(legendtext)
axis('tight')
fl.show(thisjobname()+'.pdf')
\end{lstlisting}
\end{tiny}
}
\hidden{

\subsubsection{Saturation curve following ESR}\hidden{\maxtime{0.5}
\paragraph{copy previous parameters}
Copy experiment just run + change number of scans to 1 + decrease resolution along $x$.

Decrease receiver gain by an order of magnitude.

Run quick saturation with 3,6,10.

Save quicksat parameters.

\fn{hydroxytempo_50uM_quicksat_110114}
Blue stop once it is on the decrease.

Copy experiment.

Set receiver gain with box.

Also set number of scans appropriate for the concentration.

Start at 3 $dB$ and go to 30 $dB$, or 40 $dB$ for low concentrations, with the step size equal to $(dB\;width)*t_{scan}*n_{scans}/420$ (scan time for 7 minutes).

Divide the span by the stepsize to get the ``resolution along $y$.''

Start + click to stop so field doesn't run through any signal when returning to start.

Set watch timer for time.

\paragraph{process the data}
Save data.

\fn{hydroxytempo_50uM_sat_110114}
Close quicksat experiment, and save.

Wait for it to finish, and close and save.

Select all, and winscp.

Just leave it alone, since I know the data's good, and instead just plot an image to show the difference in snr

Set the smoothing to half the linewidth (about 0.25), which should provide optimal SNR, then optimize the threshold



\begin{tiny}
\hidden{
\begin{lstlisting}
scaling = 50.24/(10**(-6.0/10.0))
setting = r_[3:34+1:1]
power = scaling*(10**(-0.1*setting))
esr_saturation(DATADIR+'cnsi_data/oxotempo_50uM_saturation_100427',power,threshold=0.6,smoothing=0.25)
print '\n\nsettings:',setting
\end{lstlisting}


\begin{lstlisting}
data = load_file(DATADIR+'cnsi_data/oxotempo_50uM_saturation_100427',dimname='power')
image(data)
title('new saturation data')
lplot('newsat'+thisjobname()+'.pdf')
data = load_file(DATADIR+'cnsi_data/oxotempo_50uM_saturation_coax_100309',dimname='power')
image(data)
title('old saturation data')
lplot('oldsat'+thisjobname()+'.pdf')
\end{lstlisting}
}
\end{tiny}


}
}
\subsubsection{set up DNP}\maxminutes{11}
\paragraph{insert probe and autotune}
Put into tune mode before removing sample, and turn down power.

Turn off field.

Check marker marks on probe.

Insert probe with sample.

Attach tuning box, then position and tighted screw.

Turn up air to 10 SCFH.

Press autotune.

\na{Detach mod coil and hang in the plastic loop (here so that I don't potentially mess up microwave tune later!).}

\paragraph{Tune and set resonant field}
Copy nmr template dataset.
\fn{dopc_32mM_3p_120123}

Overwrite exp 1 with desired parameters.
\fn{dopc_dil_3p_120118}

Round nearest 0.0005 $GHz$ to get YIG frequency.

Open first experiment and set NMR sfo1 to (1.5167)*(YIG frequency).

Tune NMR {\tt (wobb)}.

\precaution{Run jf\_dnpconf if necessary -- specifically may need to decrease my attenuation settings by 2.7 $dB$ (or just round it to 3 $dB$).}

Tune microwave by hand
{\small Simultaneously set:
\begin{itemize}
    \item bias (diode centered at 50 $dB$)
    \item frequency (AFC)
    \item phase (coarse: dip in tune mode; fine: furthest right diode w.r.t.signal phase while AFC remains locked)
\end{itemize}
up to 5 $dB$.}
\begin{lstlisting}
calcfielddata(9.789948,'oxotempo','cnsi')
\end{lstlisting}

Check that YIG frequency is still good

\na{Raise box.}

\na{Double check EPR tune.}

EPR to standby.

Set field.

\paragraph{determine ratio (if desired)}
\hidden{ Set sfo1 to resonance.  Increase aq to 2.0.  jf\_zgm and lightning bolt for o1.  Copy microwave frequency from ESR and sfo1 from center frequency.  
\begin{lstlisting}
microwave_frequency = 9.344975 #(from file)
field = 3329.828
nmr_frequency = 14.1730638
field = field * 1e-4
chemical = 'hydroxytempo'
obs('previous:')
calcfielddata(microwave_frequency,chemical,'cnsi')
obs('new:')
save_data({chemical+'_elratiocnsi':microwave_frequency/field})
save_data({chemical+'_nmrelratio':nmr_frequency/microwave_frequency})
calcfielddata(microwave_frequency,chemical,'cnsi')
\end{lstlisting}
Set aq back to 0.02.  }
\paragraph{ Match NMR + YIG frequencies}
jf\_zg $\Rightarrow$ for 3.5 $\mu L$ sample, peak should 10-15 high.

Check that the source is on, and reads 0.26 $A$ (rather than 0.54 $A$).

Click lightning bolt to set o1 (while zoomed in with top window).

Set YIG frequency -- and optionally the $ppt$ value -- and test amp switch with jf\_setmw (leave other parameters as default).

Iteratively adjust the field.
{\small
\begin{enumerate}
	\item jf\_setmw (31.5 $dB$, YIG frequency),
	\item then if field offset is significant, adjust $B_0$, otherwise stop
	\item jf\_zgm
	\item then set sfo1 to NMR resonance (lightning bolt) + start over
\end{enumerate}
}

\paragraph{90 time and ready amp}
jf\_zg + zoom + dpl1 + run paropt (p1,8,1,3)

\precaution{This should go down to up at the full cycle;
if problems, (p1,1,1,11).}
\precaution{If you want to decrease the narrow noise spikes, minimize WinEPR + turn off bridge + put all bridge cables on top of dewar while paropt is running; could be obsolete with box?}
Flip waveguide switch.

Set + record $t_{90}$ (``p1'')= $\mu s$ in full cycle / 4 (At this point, you've determined the resonance frequency, resonant field, and $t_{90}$).

\subsubsection{Start DNP}\maxminutes{12}
\paragraph{Estimate $T_1$ (with heating)}

\hidden{
If previous $T_1$ data for this sample is available, first estimate the $T_1$ based on the spin label concentration.

\begin{lstlisting}
# edit following
oldconc = 2e-3 # set this to None if oldt1 is given at the same concentration
oldt1 = 0.89
newconc = 10e-6
# leave the following alone
def estimate_t1(concentration,value,newconc):
    water = 1./2.6
    relaxivity = 1./value
    relaxivity -= water
    relaxivity /= concentration
    relaxivity *= newconc
    relaxivity += water
    return 1./relaxivity
def estimate_hot_t1(thist1):
    water = 1./2.6
    hot_water = 1./3.65 # this is for the new ``closed'' type probe # 7/8 adjustted this
    thist1 = 1./thist1 # convert to a rate
    thist1 -= water # figure out which part is from water
    thist1 += hot_water # add back in for heating
    return 1./thist1
if oldconc == None:
    t1 = oldt1
else:
    t1 = estimate_t1(oldconc,oldt1,newconc)
    obs(dp(t1,2),r' $s$ without heating')
obs(dp(estimate_hot_t1(t1),2),r' $s$ with heating')
\end{lstlisting}
}
Then, if necessary, run jf\_t10 with default parameters (should be min 1,max 2.5,steps 5,min ratio 0.08 (should have been 0.05 to match), max ratio 4, pull from exp 4, put in exp 101).


Copy the nmr data.

Process the $T_1$


\begin{lstlisting}
# these change
name = 'dopc_dil_3p_120123'
path = DATADIR+'cnsi_data/'
# alter these
dnp_for_rho(path,name,[],expno=[],t1expnos = [101],
        integration_width = 150,peak_within = 500,
        show_t1_raw = True,phnum = [4],phchannel = [-1],
        h5file='t1_estimation_only.h5',
        clear_nodes=True)
t1 = retrieve_T1series('t1_estimation_only.h5',name)
def estimate_hot_t1(thist1):
    water = 1./2.6
    hot_water = 1./3.8 # this is for the new ``closed'' type probe # 7/8 adjustted this
    thist1 = 1./thist1 # convert to a rate
    thist1 -= water # figure out which part is from water
    thist1 += hot_water # add back in for heating
    return 1./thist1
obs(r'Min $T_1\approx$',lsafe(t1['power',0]),r'\quad $T_{1,max}\approx$',lsafe(estimate_hot_t1(t1)),r' $s$ with heating')
\end{lstlisting}
\paragraph{run DNP}
Start jf\_dnp: Set minimum and maximum $T_1$ values based on $T_1$ estimate (with heating).

Set number of $T_1$ time to closer to 20 $min$ for lower concentrations and closer to 12 $min$ for higher concentrations.

Set watch timer for experiment time.

Check the lock voltage.

\subsubsection{wait for DNP to run}\maxtime{1.2}

\subsubsection{process DNP + put back system}\maxminutes{5}\label{sec:concissues_dopclowconc_process}
\paragraph{process}
Transfer NMR files.

Change chemical names, concentration, and run number.

\na{Change the $T_{1,0}$ chemical name (and concentration?).}

{\bf why does the $T_{1,0}$ have a concentration, since it will always be 0?}
\na{Uncomment search delete datanodes.}

Mask/unmask $T_1$'s as necessary.

Run processing.

Check that my longest $T_1$ falls within range.

{\bf note that here, I really should be able to get the code to check the longest $T_1$ against cnst 9 in exp 1}

\na{Add to the compilation, to see how the data looks.}

Check the consistency of the enhancements with decreasing power.

Comment ``search delete datanode.''

{\bf should actually be able to get rid of the guessing problem by switching it to guess with the pseudoinverse!}

\begin{tiny}
\begin{lstlisting}
import textwrap
name = 'dopc_32mM_3p_120123'
path = DATADIR+'cnsi_data/'
search_delete_datanode('dnp.h5',name)
# leave the rest of the code relatively consistent
#{{{ generate the powers for the T1 series
print 'First, check the $T_1$ powers:\n\n'
fl = []
t1_dbm,fl = auto_steps(path+name+'/t1_powers.mat',threshold = -35,t_minlength = 1.5*60,t_maxlen = 40*60,t_start = 4.9*60.,t_stop = inf,first_figure = fl)
print r't1\_dbm is:',lsafen(t1_dbm)
lplotfigures(fl,'t1series_'+name)
print '\n\n'
t1mask = bool8(ones(len(t1_dbm)))
# the next line will turn off select (noisy T1 outputs)
# enter the number of the scan to remove -- don't include power off
#t1mask[-1] = 0
#}}}
dnp_for_rho(path,name,integration_width = 160,peak_within = 500,
        show_t1_raw = True,phnum = [4],phchannel = [-1],
        t1_autovals = r_[2:2+len(t1_dbm)][t1mask],
        t1_powers = r_[t1_dbm[t1mask],-999.],
        power_file = name+'/power.mat',t_start = 4.6,
        extra_time = 12.,
        chemical = 'dopc_32mM', concentration = 984e-6,
        dontfit = True, run_number = 120117)
standard_noise_comparison(name)
\end{lstlisting}
\end{tiny}

\paragraph{put back system}
\na{Delete most recent EPR, so others don't screw them up.}

Field off.

\na{Hook up mod coil.}

Flip back ESR switch.

Turn off the air + check rate.

Pull out + check sample.

SVN, then copy working copy of notebook into compilation.

\na{Add to current summary.}


Clear + copy to protocol.

\timeblockend
If done, remove air tube.

Unscrew top collet + carry box w/ dewar around to back.

Remove glass tube and replace dewar.

Insert inner collet.

Screw down top collet holder.

Cap cavity.

Tell software that bridge is on + switch to tune mode

Check for dip near \sout{9.88~GHz}\add{9.31~GHz}.

Autotune + record Q.
Remove tuning box.

If done, magnet off, ESR off, chiller off, source output off.

