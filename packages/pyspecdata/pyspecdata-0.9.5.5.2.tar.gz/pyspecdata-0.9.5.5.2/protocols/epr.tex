\subsection{Run EPR of all POPE-M samples}\maxtime{2.73}\label{sec:vesicles_byme1_redoeprall_m}
\timeblockstart
\timeblocktotal{2.73}
\subsubsection{Sample prep}
Walk over.

Prep samples.


\subsubsection{Calibrate position on new EPR tube}
\hidden{
Now, I need to calibrate the position.

Start by opening the position experiment from before (which uses only 4 scans).
\fn{calpos_4at_pos0_110503.par}
Tune empty cavity.

Be sure tube is marked.

\paragraph{position -3}
Tune mode.

Copy EPR file.

Insert to desired position.


Fine tune at 5 $dB$ (since zero is now floating).

Record Q.

Check tune.

\precaution{Run a quick saturation (saved ``quick sat'') to find the max, and check that I'm not saturating at the 15 $dB$ power level I'm using.}

Run the actual EPR.

Save and copy EPR files.
\fn{calpos_4at_posn2_110504.par}

Decide which way to move based on SNR.

\paragraph{position -2}
Tune mode.

Copy EPR file.

Insert to desired position.

Fine tune at 5 $dB$ (since zero is now floating).

Record Q.

Check tune.

\precaution{Run a quick saturation (saved ``quick sat'') to find the max, and check that I'm not saturating at the 15 $dB$ power level I'm using.}

Run the actual EPR.

Save and copy EPR files.
\fn{calpos_4at_posn3_110504.par}

Decide which way to move based on SNR.

\paragraph{position 0}
Tune mode.

Copy EPR file.

Insert to desired position.

Fine tune at 5 $dB$ (since zero is now floating).

Record Q.

Check tune.

\precaution{Run a quick saturation (saved ``quick sat'') to find the max, and check that I'm not saturating at the 15 $dB$ power level I'm using.}

Run the actual EPR.

Save and copy EPR files.
\fn{calpos_4at_pos0_110504.par}

Decide which way to move based on SNR.

\paragraph{position -1}
Tune mode.

Copy EPR file.

Insert to desired position.

Fine tune at 5 $dB$ (since zero is now floating).

Record Q.

Check tune.

\precaution{Run a quick saturation (saved ``quick sat'') to find the max, and check that I'm not saturating at the 15 $dB$ power level I'm using.}

Run the actual EPR.

Save and copy EPR files.
\fn{calpos_4at_posn1_110504.par}

Decide which way to move based on SNR.

\paragraph{process}

\begin{tiny}
\begin{python}
dir = '/home/franck/data/cnsi_data'
files = ['calpos_4at_posn3_110504','calpos_4at_posn2_110504','calpos_4at_pos0_110504','calpos_4at_posn1_110504']
positions = 2.0*r_[-3,-2,0,-1]+107.0
snrdata = []
normalize_field = True
find_maxslope = True
normalize_peak = True
# the next part should not change, and should be compiled into a function later
dir = dirformat(dir)
legendtext = list(files)
clf()
for index,file in enumerate(files):
   data = load_indiv_file(dir+file)
   field = r'$B_0$'
   neworder = list(data.dimlabels)
   data.reorder([neworder.pop(neworder.index(field))]+neworder) # in case this is a saturation experiment
   data -= data.copy().run_nopop(mean,field)
   figure(1)
   v = winepr_load_acqu(dir+file)
   xaxis = data.getaxis(field)
   centerfield = None
   if normalize_field:
       xaxis /= v['MF']
       newname = r'$B_0/\nu_e$'
   elif find_maxslope:
       deriv = data.copy()
       deriv.run_nopop(diff,field)
       deriv.data[abs(data.data) > abs(data.data).max()/10] = 0 # so it doesn't give a fast slope, non-zero area
       deriv = abs(deriv)
       deriv.argmax(field)
       centerfield = mean(xaxis[int32(deriv.data)])
       xaxis -= centerfield
       newname = r'$\Delta B_0$'
   else:
       newname = field
   data.rename(field,newname)
   mask = data.getaxis(newname)
   mask = mask > mask[int32(len(mask)-len(mask)/20)]
   snr = abs(data.data).max()/std(data.data[mask])
   snrdata.append(snr)
   integral = data.copy()
   integral.data -= integral.data.mean() # baseline correct it
   integral.integrate_cumulative(newname)
   figure(2)
   plot(integral,alpha=0.5,linewidth=0.3)
   pc = plot_color_counter()
   ax = gca()
   #at = twinx()
   integral.integrate_cumulative(newname)
   #plot_color_counter(pc)
   #plot(integral,alpha=0.15,linewidth=0.3)
   axes(ax)
   figure(1)
   if normalize_peak:
      normalization = abs(data).run_nopop(max,newname)
      data /= normalization
   plot(data,alpha=0.5,linewidth=0.3)
   if centerfield != None:
      legendtext[index] += ', %0.03f $G$'%centerfield
   legendtext[index] += r', SNR %0.2g $\int\int$ %0.3g'%(snr,integral[newname,-1].data[-1])
#xtl = ax.get_xticklabels()
#at.xaxis.tick_top()
#map( (lambda x: x.set_visible(False)), xtl)
plot(data.getaxis(newname)[mask],zeros(shape(data.getaxis(newname)[mask])),'k',alpha=0.2,linewidth=10)
figure(1)
autolegend(legendtext)
axis('tight')
lplot('esr_'+thisjobname()+'.pdf')
figure(2)
autolegend(legendtext)
axis('tight')
lplot('esr_integrated'+thisjobname()+'.pdf')
figure(3)
pos = 'position / $mm$'
snrdata = nddata(array(snrdata),[len(snrdata)],[pos],axis_coords = [array(positions)])
plot(snrdata,'bo')
c,fitdata = snrdata[pos,:].polyfit(pos,order = 2)
fitl = lambda x: c[0]+c[1]*x+c[2]*x**2
ax = gca()
parabolamaxat = -0.5*c[1]/c[2]
text(0.5,0.5,r'best = %0.2f $mm$'%parabolamaxat,transform = ax.transAxes, horizontalalignment = 'center',color = 'g',alpha = 0.5)
plot(r_[parabolamaxat],r_[fitl(parabolamaxat)],'go')
axis('tight')
x = r_[ax.get_xlim()[0]:ax.get_xlim()[1]:100j]
plot(x,fitl(x),'g-')
print '\n\n'
lplot('esr_optpos'+thisjobname()+'.pdf')
\end{python}
\end{tiny}

Then, I need to measure the actual length, and add that in to the measurement above.


}
\subsubsection{sample prep for EPR}

\subsubsection{EPR for MDS}\maxtime{0.116667}
\paragraph{First time only: Set up instrument}
Turn on chiller, then EPR console (magnet off).

Be sure  mod coil is hooked up, and the waveguide switch is set to ESR.

Turn on the air - 1 bottom w/ bottom of cavity taped.

Wait until bridge stops flashing, then open WinEPR + switch to tune mode.

Be sure that we have dewar $\Rightarrow$ inner white collet $\Rightarrow$ outer white piece $\Rightarrow$ o-ring $\Rightarrow$ collet $\Rightarrow$ black piece\john{I put this first because you don't want the EPR on while you adjust it.}

\precaution{Be sure frequency is at 9.34 $GHz$ (w/ EPR tube in).}

Slide white collet to correct position (one notch below red).

\paragraph{load sample}
Put into tune mode before changing samples.

Load sample with white sample holder into special EPR tube.

\precaution{Check the orientation of the collet ring from the top, to be sure it screws on straight.}
\twooptions{ Autotune ({\it easier}) at 10-12 $dB$ } { Tune by hand ({\it faster} -- simultaneously set bias (diode centered at 50 $dB$), frequency (AFC) phase (coarse: dip in tune mode; fine: furthest right diode w.r.t.signal phase while AFC remains locked)) up to 5 $dB$.}

Switch to 10 $dB$ + record $Q$.

Switch to operate straight from tune + adjust bias at 50.

Fine tune at 5 $dB$ + check going back down to 50 $dB$.

\precaution{Check jitter and float at 0 $dB$.}
First exp only: Run ``quick sat.par'' to check that it's not saturated at 15 $dB$.

Open ESR parameter set similar to this one.
\fn{popcg_4mM_5p_16dsa_epr_110419.par}

First exp only: Turn on field.

Copy experiment.

First exp only: Run.

First exp only: Only if the experiment says ``uncalibrated'' at the top $\Rightarrow$ ``I'' (interactive spectrometer control) icon, click calibrated, then set parameters to spectrum, then window.
First exp only: Stop at second peak.
First exp only: Check that modulation amplitude is $<$ 0.2 x smallest feature.
Set RG with box.

First exp only: Check that resolution along $x$ is OK.

Run actual scan.

Save ESR.
\fn{popem_4mM_5p_16dsa_epr_110517}





Cntrl-S after scan is finished.

If repeating, copy procedure and ctr-d t only.

\subsubsection{EPR for MTE}\maxtime{0.116667}
\paragraph{First time only: Set up instrument}
\hidden{
Turn on chiller, then EPR console (magnet off).

Be sure  mod coil is hooked up, and the waveguide switch is set to ESR.

Turn on the air - 1 bottom w/ bottom of cavity taped.

Wait until bridge stops flashing, then open WinEPR + switch to tune mode.

Be sure that we have dewar $\Rightarrow$ inner white collet $\Rightarrow$ outer white piece $\Rightarrow$ o-ring $\Rightarrow$ collet $\Rightarrow$ black piece\john{I put this first because you don't want the EPR on while you adjust it.}

\precaution{Be sure frequency is at 9.34 $GHz$ (w/ EPR tube in).}

Slide white collet to correct position (one notch below red).

}
\paragraph{load sample}
Put into tune mode before changing samples.

Load sample with white sample holder into special EPR tube.

\precaution{Check the orientation of the collet ring from the top, to be sure it screws on straight.}
\twooptions{ Autotune ({\it easier}) at 10-12 $dB$ } { Tune by hand ({\it faster} -- simultaneously set bias (diode centered at 50 $dB$), frequency (AFC) phase (coarse: dip in tune mode; fine: furthest right diode w.r.t.signal phase while AFC remains locked)) up to 5 $dB$.}

Switch to 10 $dB$ + record $Q$.

Adjust bias to recenter.

Fine tune at 5 $dB$ + check going back down to 50 $dB$.

\precaution{Check jitter and float at 0 $dB$.}
First exp only: Run ``quick sat.par'' to check that it's not saturated at 15 $dB$.

Open ESR parameter set similar to this one.
\fn{popem_4mM_5p_16dsa_epr_110517.par} and cut ns in half.

First exp only: Turn on field.

Copy experiment.

\hidden{First exp only: Run.

First exp only: Only if the experiment says ``uncalibrated'' at the top $\Rightarrow$ ``I'' (interactive spectrometer control) icon, click calibrated, then set parameters to spectrum, then window.
First exp only: Stop at second peak.
First exp only: Check that modulation amplitude is $<$ 0.2 x smallest feature.
Set RG with box.

First exp only: Check that resolution along $x$ is OK.
}
Run actual scan.

Save ESR.
\fn{popem_4mM_5p_pct_epr_110517}

Cntrl-S after scan is finished.

If repeating, copy procedure and ctr-d t only.

\subsubsection{EPR for MSL}\maxtime{0.116667}
\paragraph{First time only: Set up instrument}
\hidden{
Turn on chiller, then EPR console (magnet off).

Be sure  mod coil is hooked up, and the waveguide switch is set to ESR.

Turn on the air - 1 bottom w/ bottom of cavity taped.

Wait until bridge stops flashing, then open WinEPR + switch to tune mode.

Be sure that we have dewar $\Rightarrow$ inner white collet $\Rightarrow$ outer white piece $\Rightarrow$ o-ring $\Rightarrow$ collet $\Rightarrow$ black piece\john{I put this first because you don't want the EPR on while you adjust it.}

\precaution{Be sure frequency is at 9.34 $GHz$ (w/ EPR tube in).}

Slide white collet to correct position (one notch below red).

}
\paragraph{load sample}
Put into tune mode before changing samples.

Load sample with white sample holder into special EPR tube.

\precaution{Check the orientation of the collet ring from the top, to be sure it screws on straight.}
\twooptions{ Autotune ({\it easier}) at 10-12 $dB$ } { Tune by hand ({\it faster} -- simultaneously set bias (diode centered at 50 $dB$), frequency (AFC) phase (coarse: dip in tune mode; fine: furthest right diode w.r.t.signal phase while AFC remains locked)) up to 5 $dB$.}

Switch to 10 $dB$ + record $Q$.

Adjust bias to recenter.

Fine tune at 5 $dB$ + check going back down to 50 $dB$.

\precaution{Check jitter and float at 0 $dB$.}
First exp only: Run ``quick sat.par'' to check that it's not saturated at 15 $dB$.

Open ESR parameter set similar to this one.
\fn{popem_4mM_5p_pct_epr_110517.par}

First exp only: Turn on field.

Copy experiment.

\hidden{First exp only: Run.

First exp only: Only if the experiment says ``uncalibrated'' at the top $\Rightarrow$ ``I'' (interactive spectrometer control) icon, click calibrated, then set parameters to spectrum, then window.
First exp only: Stop at second peak.
First exp only: Check that modulation amplitude is $<$ 0.2 x smallest feature.
Set RG with box.

First exp only: Check that resolution along $x$ is OK.
}
Run actual scan.

Save ESR.
\fn{popem_4mM_5p_mmsl_epr_110517}

Cntrl-S after scan is finished.

If repeating, copy procedure and ctr-d t only.

\subsubsection{EPR for MLL}\maxtime{0.116667}
\paragraph{First time only: Set up instrument}
\hidden{
Turn on chiller, then EPR console (magnet off).

Be sure  mod coil is hooked up, and the waveguide switch is set to ESR.

Turn on the air - 1 bottom w/ bottom of cavity taped.

Wait until bridge stops flashing, then open WinEPR + switch to tune mode.

Be sure that we have dewar $\Rightarrow$ inner white collet $\Rightarrow$ outer white piece $\Rightarrow$ o-ring $\Rightarrow$ collet $\Rightarrow$ black piece\john{I put this first because you don't want the EPR on while you adjust it.}

\precaution{Be sure frequency is at 9.34 $GHz$ (w/ EPR tube in).}

Slide white collet to correct position (one notch below red).

}
\paragraph{load sample}
Put into tune mode before changing samples.

Load sample with white sample holder into special EPR tube.

\precaution{Check the orientation of the collet ring from the top, to be sure it screws on straight.}
\twooptions{ Autotune ({\it easier}) at 10-12 $dB$ } { Tune by hand ({\it faster} -- simultaneously set bias (diode centered at 50 $dB$), frequency (AFC) phase (coarse: dip in tune mode; fine: furthest right diode w.r.t.signal phase while AFC remains locked)) up to 5 $dB$.}

Switch to 10 $dB$ + record $Q$.

Adjust bias to recenter.

Fine tune at 5 $dB$ + check going back down to 50 $dB$.

\precaution{Check jitter and float at 0 $dB$.}
First exp only: Run ``quick sat.par'' to check that it's not saturated at 15 $dB$.

Open ESR parameter set similar to this one.
\fn{popem_4mM_5p_mmsl_epr_110517.par}

First exp only: Turn on field.

Copy experiment.

\hidden{First exp only: Run.

First exp only: Only if the experiment says ``uncalibrated'' at the top $\Rightarrow$ ``I'' (interactive spectrometer control) icon, click calibrated, then set parameters to spectrum, then window.
First exp only: Stop at second peak.
First exp only: Check that modulation amplitude is $<$ 0.2 x smallest feature.
Set RG with box.

First exp only: Check that resolution along $x$ is OK.
}
Run actual scan.

Save ESR.
\fn{popem_4mM_5p_mmll_epr_110517}

Cntrl-S after scan is finished.

If repeating, copy procedure and ctr-d t only.

\subsubsection{(debug) code for test gain}
Hit Cntrl-A for ssh transfer.

Process.



\begin{tiny}
\begin{python}
dir = '/home/franck/data/cnsi_data'
files = ['test_gain1','test_gain2','test_gain3','test_gain4','test_gain5','test_gain6']
normalize_field = True
find_maxslope = True
subtract_first = False
normalize_peak = False
# the next part should not change, and should be compiled into a function later
dir = dirformat(dir)
fl = figlistl()
if subtract_first:
    firstdata = load_indiv_file(dir+files.pop(0))
legendtext = list(files)
for index,file in enumerate(files):
   data = load_indiv_file(dir+file)
   if subtract_first:
       data -= firstdata
   field = r'$B_0$'
   neworder = list(data.dimlabels)
   data.reorder([neworder.pop(neworder.index(field))]+neworder) # in case this is a saturation experiment
   data -= data.copy().run_nopop(mean,field)
   fl.next('epr')
   v = winepr_load_acqu(dir+file)
   if index == 0:
        fieldbar = data[field,lambda x: logical_and(x>x.mean(),x<x.mean()+10.)]
        fieldbar.data[:] = 0.5
        fieldbar.data[0] = 0.6
        fieldbar.data[-1] = 0.6
        fxaxis = fieldbar.getaxis(field)
   xaxis = data.getaxis(field)
   centerfield = None
   if normalize_field:
       xaxis /= v['MF']
       if index == 0:
           fxaxis /= v['MF']
       newname = r'$B_0/\nu_e$'
   elif find_maxslope:
       deriv = data.copy()
       deriv.run_nopop(diff,field)
       deriv.data[abs(data.data) > abs(data.data).max()/10] = 0 # so it doesn't give a fast slope, non-zero area
       deriv = abs(deriv)
       deriv.argmax(field)
       centerfield = mean(xaxis[int32(deriv.data)])
       xaxis -= centerfield
       if index == 0:
           fxaxis -= centerfield
       newname = r'$\Delta B_0$'
   else:
       newname = field
   data.rename(field,newname)
   if index == 0:
       fieldbar.rename(field,newname)
   mask = data.getaxis(newname)
   mask = mask > mask[int32(len(mask)-len(mask)/20)]
   snr = abs(data.data).max()/std(data.data[mask])
   integral = data.copy()
   integral.data -= integral.data.mean() # baseline correct it
   integral.integrate_cumulative(newname)
   fl.next('epr_int')
   plot(integral,alpha=0.5,linewidth=0.3)
   pc = plot_color_counter()
   integral.integrate_cumulative(newname)
   fl.next('epr')
   if normalize_peak:
      normalization = abs(data).run_nopop(max,newname)
      data /= normalization
   ax = gca()
   plot(data+array(ax.get_ylim()).min(),alpha=0.5,linewidth=0.3)
   axis('tight')
   if index == 0:
       fieldbar *= array(ax.get_ylim()).max()
   if centerfield != None:
      legendtext[index] += ', %0.03f $G$'%centerfield
   legendtext[index] += r', SNR %0.2g $\int\int$ %0.3g'%(snr,integral[newname,-1].data[-1])
#xtl = ax.get_xticklabels()
#at.xaxis.tick_top()
#map( (lambda x: x.set_visible(False)), xtl)
plot(data.getaxis(newname)[mask],zeros(shape(data.getaxis(newname)[mask])),'k',alpha=0.2,linewidth=10)
fl.next('epr')
plot(fieldbar,'k',linewidth = 2.0)
#autolegend(legendtext)
axis('tight')
fl.next('epr_int')
autolegend(legendtext)
axis('tight')
fl.show(thisjobname()+'.pdf')
\end{python}
\end{tiny}




\subsubsection{code for all above}
Hit Cntrl-A for ssh transfer.

Process.

When done, copy working copy into main file, svn, then wipe working copy and move into protocol dir.

\begin{tiny}
\begin{python}
dir = '/home/franck/data/cnsi_data'
files = ['popeca_4mM_5p_cnl_epr_110504','popem_4mM_5p_mmll_epr_110517','popem_4mM_5p_mmsl_epr_110517','popem_4mM_5p_pct_epr_110517','popem_4mM_5p_16dsa_epr_110517']
normalize_field = True
find_maxslope = True
subtract_first = True
normalize_peak = False
# the next part should not change, and should be compiled into a function later
dir = dirformat(dir)
fl = figlistl()
if subtract_first:
    firstdata = load_indiv_file(dir+files.pop(0))
legendtext = list(files)
for index,file in enumerate(files):
   data = load_indiv_file(dir+file)
   if subtract_first:
       data -= firstdata
   field = r'$B_0$'
   neworder = list(data.dimlabels)
   data.reorder([neworder.pop(neworder.index(field))]+neworder) # in case this is a saturation experiment
   data -= data.copy().run_nopop(mean,field)
   fl.next('epr')
   v = winepr_load_acqu(dir+file)
   if index == 0:
        fieldbar = data[field,lambda x: logical_and(x>x.mean(),x<x.mean()+10.)]
        fieldbar.data[:] = 0.5
        fieldbar.data[0] = 0.6
        fieldbar.data[-1] = 0.6
        fxaxis = fieldbar.getaxis(field)
   xaxis = data.getaxis(field)
   centerfield = None
   if normalize_field:
       xaxis /= v['MF']
       if index == 0:
           fxaxis /= v['MF']
       newname = r'$B_0/\nu_e$'
   elif find_maxslope:
       deriv = data.copy()
       deriv.run_nopop(diff,field)
       deriv.data[abs(data.data) > abs(data.data).max()/10] = 0 # so it doesn't give a fast slope, non-zero area
       deriv = abs(deriv)
       deriv.argmax(field)
       centerfield = mean(xaxis[int32(deriv.data)])
       xaxis -= centerfield
       if index == 0:
           fxaxis -= centerfield
       newname = r'$\Delta B_0$'
   else:
       newname = field
   data.rename(field,newname)
   if index == 0:
       fieldbar.rename(field,newname)
   mask = data.getaxis(newname)
   mask = mask > mask[int32(len(mask)-len(mask)/20)]
   snr = abs(data.data).max()/std(data.data[mask])
   integral = data.copy()
   integral.data -= integral.data.mean() # baseline correct it
   integral.integrate_cumulative(newname)
   fl.next('epr_int')
   plot(integral,alpha=0.5,linewidth=0.3)
   pc = plot_color_counter()
   integral.integrate_cumulative(newname)
   fl.next('epr')
   if normalize_peak:
      normalization = abs(data).run_nopop(max,newname)
      data /= normalization
   ax = gca()
   plot(data+array(ax.get_ylim()).min(),alpha=0.5,linewidth=0.3)
   axis('tight')
   if index == 0:
       fieldbar *= array(ax.get_ylim()).max()
   if centerfield != None:
      legendtext[index] += ', %0.03f $G$'%centerfield
   legendtext[index] += r', SNR %0.2g $\int\int$ %0.3g'%(snr,integral[newname,-1].data[-1])
#xtl = ax.get_xticklabels()
#at.xaxis.tick_top()
#map( (lambda x: x.set_visible(False)), xtl)
plot(data.getaxis(newname)[mask],zeros(shape(data.getaxis(newname)[mask])),'k',alpha=0.2,linewidth=10)
fl.next('epr')
plot(fieldbar,'k',linewidth = 2.0)
#autolegend(legendtext)
axis('tight')
fl.next('epr_int')
autolegend(legendtext)
axis('tight')
fl.show(thisjobname()+'.pdf')
\end{python}
\end{tiny}




