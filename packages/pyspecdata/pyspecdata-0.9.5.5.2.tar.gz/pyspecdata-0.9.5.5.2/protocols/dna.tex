\subsection{CS14 DNA unbound}\label{sec:dna_cs14_unbounddna}
\timeblockstart
\timeblocktotal{3.6}
Here, we are looking at the DNP on CS14, which Katherine looked at before.
This is the part that has the ligand, but is not bound to streptavidin.

\subsubsection{initial setup}\maxminutes{57}
\paragraph{First Time}
Be sure to carry over:
\begin{itemize}
    \item microcentrifuge tubes
    \item toolbox 
    \item multimeter 
    \item samples (leave in my box in CNSI fridge)
    \item magnifying glasses
\end{itemize}

Attach air tube to temperature control, insert, turn on temp control.

\paragraph{always}
Be sure to change section title and label!!!

Copy section label to list for the day.

$B_0$ field off, air off.

\paragraph{capillary prep}
\subparagraph{using the sealed tubes from Yuan}

Cut a decent length of tube.

Load with GELoader pipette

Centrifuge sample down at 1000 rpm.

Cut again, then add critoseal and withdraw air with pipette

Put stuff back in fridge!

Measure sample.

Insert in probe.

\subsubsection{Quick ESR in probe}\maxminutes{11}
Put into tune mode, $40\;dB$ first.

Insert probe, and turn up air to 12~SCFH (since a new sample, just use a higher flow rate).

Move to near 9.77 and autotune.

\paragraph{test for signal}
try same parameters as her.

try to increase the modulation amplitude, just to see signal.

zoom in and repeat.

check for saturation 15, down by 3~dB.

save saturated scan.
\fn{dna_cs14_overmod_sat_120314}

use a power slightly before saturation.

Make an experiment with this power, but turn down the modulation amplitude and change the receiver window and the resolution and number of scans.

Increase the modulation amplitude to 2~G.

\paragraph{actual ESR}
Turn on field.

Copy experiment.
\fn{dna_cs14_overmod_120314}

\paragraph{always}
Run actual scan.

Save ESR.
\fn{dna_cs14_bound_overmod_120314}

Cntrl-S after scan is finished.

Hit Cntrl-A for ssh transfer and process.

\begin{tiny}
\begin{lstlisting}
fl = figlistl()
standard_epr(dir = DATADIR+'cnsi_data',
    files = ['dna_cs14_overmod_120314',
        'dna_cs14_bound_overmod_120314',
        'dna_cs14_nl_bound_overmod_120314',
        'dna_cs14_unbound_overmod_120315'],
    figure_list = fl)
fl.show(thisjobname()+'.pdf')
\end{lstlisting}
\end{tiny}

\paragraph{determine ratio}
\ntd{I don't have git set up right here, but later, I should paste the ratio procedure here}
\subsubsection{DNP Experimental Setup}

\paragraph{tune NMR, when I'm not using microwave power}
Set switch to DNP amp, set bridge to standby.

Attach probe w/ coil perpendicular to $B_0$ + at correct height, then turn on air (14~SCFM).

Turn on $B_0$ field.

Make a new NMR experiment based on template, and change into that experiment.

\texttt{wobb}. \adlin{back moves lf/right front moves down/up}

\texttt{gs} and field sweep $B_0$ over about 0.1~G until 0 (set center field, then copy to static field, then back to time scan).

Check that the YIG's DC power supply reads 0.26-0.27~A with first light on and other lights green.

\paragraph{tune and set resonant field}
Attach probe w/ coil perpendicular to $B_0$ + at correct height, then turn on air (20~SCFM).

Put into tune mode before inserting sample, and turn down power.

Move to near 9.77 and autotune + turn on $B_0$ field.

Make a new NMR experiment based on template, and change into that experiment.

\fn{dopc_984mM_120503}
\texttt{wobb}. \adlin{back moves lf/right front moves down/up}

Check the ESR tune by hand up to 0~dB and record microwave frequency.
\begin{python}[off]
calcfielddata(9.790094,'mtsl','cnsi')
\end{python}
Put ESR bridge in standby mode and disconnect the mod coil.

Set the field to a reasonable value (last value usually fine, if not use the value from the ratio).

Use \texttt{jf\_setmw} to set frequency ratio and YIG frequency.
Check that the YIG's DC power supply reads 0.26-0.27~A with first light on and other lights green.

\texttt{jf\_zg} for signal about 15 high, then set it on resonance (lightning bolt o1).

Set sfo1 to (ppt value)*(YIG frequency), then use jf\_setmw to center resonance in gs.

\paragraph{calibrate 90 time and $T_1$}

Run \texttt{jf_zg}; zoom in; \texttt{dpl1}; then \texttt{paropt} with \texttt{p1},8,1,3 which gives signal at 8\us, 9\us, and 10\us pulse length, which should be $\approx 360^o$.

Flip the waveguide switch so the amp is connected (not the bridge).

If needed, run a rough experiment to determine the $T_1$ time, and put it in experiment 101.

\begin{scriptsize}
\begin{python}[off]
# these change
name = 'dna_cs19_bound_120419' # the name of the experiment directory
path = DATADIR+'franck_cnsi/nmr/' # the name of the directory where you are storing your data
# the following stays the same
dnp_for_rho(path,name,[],expno=[],t1expnos = [101],
        integration_width = 150,peak_within = 500,
        show_t1_raw = True,phnum = [4],
        phchannel = [-1],
        h5file='t1_estimation_only.h5',
        pdfstring = name,
        clear_nodes=True)
t1 = retrieve_T1series('t1_estimation_only.h5',name)
def estimate_hot_t1(thist1):
    water = 1./2.6
    hot_water = 1./4.2 # this is for the new ``closed''
    #type probe # 7/8 adjustted this
    thist1 = 1./thist1 # convert to a rate
    thist1 -= water # figure out which part is from
    #water
    thist1 += hot_water # add back in for heating
    return 1./thist1
obs(r'Min $T_1\approx$',lsafe(t1['power',0]),r'\quad $T_{1,max}\approx$',
        lsafe(estimate_hot_t1(t1)),
        r' $s$ with heating')
\end{python}
\end{scriptsize}
\paragraph{determine $T_1(t)$}
Use jf\_dnp to determine number of points for a scan \sout{17~min}12~min long:
tau187 nl $\Rightarrow$ min of 2.1 max of 2.2
DNA CS19 unbound $\Rightarrow$ min of 2.1 max of 2.3
DNA CS19 bound $\Rightarrow$ min of 2.0 max of 2.2, 16 steps

Run jf\_t10s:
Set 35 experiments at 17~min for 9h55min, exp 4$\Rightarrow$ 501.

Come back after it ran.

\paragraph{run DNP}
Start jf\_dnp, and write down $T_1$ times entered.
bound CS19 $\Rightarrow$ 1.5 to 2.1 to be safe
DOPC high concentration $\Rightarrow$ 0.67 to 0.8 to be safe

\subsubsection{Processing}

\paragraph{process DNP}
Process results.


\begin{scriptsize}
\begin{python}[off]
import textwrap # don't pay attention to this
# change the following parameters
name = 'enterexperiment' # replace with the name of the experiment
chemical_name = 'enterchemicalhere' # this is the name of the chemical i.e. 'hydroxytempo' or 'DOPC', etc.
run_number = 120418 # this is a continuous run of data
concentration = 200e-6 # the concentration of spin label
dontfit = False # only set to true where you don't expect enhancement
path = DATADIR+'franck_cnsi/nmr/'
# if after setting all these correctly,
# it complains about your T1 experiments
# try to uncomment the t1mask line below
###########################
# the following stays the same
#search_delete_datanode('dnp.h5',name)
# leave the rest of the code relatively consistent
#{{{ generate the powers for the T1 series
print 'First, check the $T_1$ powers:\n\n'
fl = []
t1_dbm,fl = auto_steps(path+name+'/t1_powers.mat',
    threshold = -35,t_minlength = 5.0*60,
    t_maxlen = 40*60, t_start = 4.9*60.,
    t_stop = inf,first_figure = fl)
print r't1\_dbm is:',lsafen(t1_dbm)
lplotfigures(fl,'t1series_'+name)
print '\n\n'
t1mask = bool8(ones(len(t1_dbm)))
# the next line will turn off select (noisy T1
# outputs) enter the number of the scan to remove --
# don't include power off
#t1mask[-1] = 0 # this is the line you sometimes want to uncomment
#}}}
dnp_for_rho(path,name,integration_width = 160,
        peak_within = 500, show_t1_raw = True,
        phnum = [4],phchannel = [-1],
        t1_autovals = r_[2:2+len(t1_dbm)][t1mask],
        t1_powers = r_[t1_dbm[t1mask],-999.],
        power_file = name+'/power.mat',t_start = 4.6,
        chemical = chemical_name,
        concentration = concentration,
        extra_time = 6.0,
        dontfit = dontfit,
        run_number = run_number,
        threshold = -50.)
standard_noise_comparison(name)
# tried to fix error on cov more fix more fix more fix more fix more fix
\end{python}
\end{scriptsize}

Check that my longest $T_1$ falls within range I used.
\ntd{just code this in}

Check the consistency of the enhancements with decreasing power.

Comment {\tt search\_delete\_datanode}.

Git commit.

Copy working\_copy into compilation for this project.

Turn $B_0$ off, remove probe + look at sample.

\paragraph{Process multiple $T_1$ data}
Copy (or Git once that's ready) files.

Change all parameters at beginning.

Run script

\begin{scriptsize}
\begin{python}[off]
# these change
name = 'dna_cs19_nl_unbound_120418' # name of dataset
chemical = 'dna_cs19' # chemical name
concentration = 0.0 # concentration of spin label (usually 0)
run_number =  120418
number_of_repeats = 30 # number of T1 experiments run
start_exp = 501 # the experiment you started at
# everything else stays the same
path = DATADIR+'franck_cnsi/nmr/'
dnp_for_rho(path,name,[],expno=[],
    t1expnos = r_[start_exp:start_exp+number_of_repeats],
    integration_width = 150,peak_within = 500,
    show_t1_raw = True,phnum = [4],
    phchannel = [-1],
    chemical = chemical,
    concentration = concentration,
    run_number = run_number,
    h5file='dnp.h5',
    pdfstring = name,
    clear_nodes = False)
t1 = retrieve_T1series('dnp.h5',
    name,
    chemical,
    concentration)
t1.rename('power','expno')
t1.labels('expno',r_[1:1+number_of_repeats])
plot(t1)
lplot(name+'_t1_vs_time.pdf')
\end{python}
\end{scriptsize}
\subsubsection{ESR}\maxminutes{11}

(Here, I run ESR after the DNP, so I don't have to worry about timing, but still get the double integral)
Put into tune mode, $40\;dB$ first.

Load sample in the teflon sample holder (I want ~77 $mm$ from collet to bottom of sample, need to remeasure) -- weight the top with the collet holder.

Move to near 9.77 and autotune.

Open ESR parameter set similar to this one.
\fn{grocomplex_120202.par}

Turn on field.

Copy experiment.

If not exactly the same type of sample, test run to check some stuff.

\precaution{Only if the experiment says ``uncalibrated'' at the top $\Rightarrow$ ``I'' (interactive spectrometer control) icon, click calibrated, then set parameters to spectrum, then window.}

\paragraph{if new type of sample}
Stop at second peak.

Check that modulation amplitude is $<$ 0.2 x smallest feature.

Set RG with box.

Check that resolution along $x$ is OK.

\paragraph{always}

Run actual scan.

Save ESR.
\fn{groes_120202}

For 8 scans, alarm for about 2.5 min.

Cntrl-S after scan is finished.

Hit Cntrl-A for ssh transfer and process.


\begin{tiny}
\begin{lstlisting}
fl = figlistl()
standard_epr(dir = DATADIR+'cnsi_data',
    files = ['grocomplex_120224','groes_120224','grocomplex_rep_120224'],
    figure_list = fl)
fl.show(thisjobname()+'.pdf')
\end{lstlisting}
\end{tiny}

Put into tune mode before removing sample, and turn down power.

Run background scan.

\subsubsection{wrap up}\maxminutes{18}
Turn off field and air and flip switch.

Pull out and measure sample.

Copy this file into compilation for the project.

Clear + copy to protocol.

\subparagraph{Run ESR background scan}
\subparagraph{Last experiment only}
Put quartz tube in loops to right, reattach mod coil.

Replace the dewar (being sure to include both inner and outer top collets).

Find dip near 9.88~GHz (if without dewar) or 9.31~GHz (if with dewar), and autotune.

Git commit again and Git push.


\timeblockend
